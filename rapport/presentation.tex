\chapter{Introduction à ShankScript}


\section{Qu'est ce que ShankScript ?}


En informatique, un langage de programmation est une notation conventionnelle destinée à formuler des algorithmes et produire des programmes informatiques qui les appliquent. D'une manière similaire à une langue naturelle, un langage de programmation est composé d'un alphabet, d'un vocabulaire, de règles de grammaire et de significations .
\\[0.5cm]
Les langages de programmation permettent de décrire d'une part les structures des données qui seront manipulées par l'appareil informatique, et d'autre part d'indiquer comment sont effectuées les manipulations, selon quels algorithmes. Ils servent de moyens de communication par lesquels le programmeur communique avec l'ordinateur, mais aussi avec d'autres programmeurs ; les programmes étant d'ordinaire écrits, lus, compris et modifiés par une équipe de programmeurs .
\\
Un langage de programmation est mis en œuvre par un traducteur automatique : compilateur ou interpréteur. Un compilateur est un programme informatique qui transforme dans un premier temps un code source écrit dans un langage de programmation donné en un code cible qui pourra être directement exécuté par un ordinateur, à savoir un programme en langage machine ou en code intermédiaire, tandis que l'interpréteur réalise cette traduction « à la volée ».
\\[0.5cm]
Les langages de programmation offrent différentes possibilités d'abstraction, et une notation proche de l'algèbre, permettant de décrire de manière concise et facile à saisir les opérations de manipulation de données et l'évolution du déroulement du programme en fonction des situations. La possibilité d'écriture abstraite libère l'esprit du programmeur d'un travail superflu, notamment de prise en compte des spécificités du matériel informatique, et lui permet ainsi de se concentrer sur des problèmes plus avancés.
\\[0.5cm]
Chaque langage de programmation supporte un ou plusieurs styles de programmation – paradigmes. Les notions propres au paradigme font partie du langage de programmation, permettant au programmeur d'exprimer dans le langage de programmation une solution qui a été imaginée selon ce paradigme.
\\
Les premiers langages de programmation ont été créés dans les années 1950. De nombreux concepts de l'informatique ont été lancés par un langage, avant d'être améliorés et étendus dans les langages suivants. La plupart du temps la conception d'un langage de programmation a été fortement influencée par l'expérience acquise avec les langages précédents.
\\[0.5cm]
Dans nôtre cas , ShankSscript est un peu ceci et cela ,avec le manque du temps qu 'on a ,et pour cette première version le langage et paradigmes ,est il est interprété ;vous pouvez valider commande par commande comme dans le cas de python sinon vous avez la possibilité d'écrire dans un fichier Bash avec l'extension ".sks" et par la suite le compiler .





\section{Premiers pas avec l'interpréteur de commandes ShankScript}


\section{Les variables}
\section{Les structures conditionnelles}
\section{Les boucles}
\section{Les chaînes de caractères}